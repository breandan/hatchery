\documentclass[12pt]{article}
\usepackage{geometry}
\usepackage{titling}
\renewcommand{\familydefault}{\sfdefault}
\setlength{\droptitle}{-7em}
\pretitle{\begin{flushleft}\textbf{Proposal Title:}\\ \LARGE}
\posttitle{\par\end{flushleft}}
\preauthor{\begin{flushleft}\textbf{Presenter(s):}\large
           \lineskip 0.5em%
           \begin{tabular}[t]{l}}
\postauthor{\end{tabular}\end{flushleft}}
\predate{}
\postdate{}
\usepackage{sectsty}
\sectionfont{\fontsize{12}{15}\selectfont}
\usepackage{nopageno}
\usepackage{titlesec}
\titlespacing{\section}{0pt}{*3}{*0}
\geometry{
  body={6.5in, 9.0in},
  left=1.0in,
  top=1.0in
}
\date{}
% Everything above this line should remain the same

\begin{document}
\title{Development tools for robotics programming}
\author{Breandan Considine (McGill University)}
\maketitle

\section*{Summary:}
Hatchery is a tool for building ROS applications based on the CLion IDE. Originally developed as an educational aide for robotics programming, it offers general-purpose support for creating ROS projects, integration with common ROS plugins, assistance for running and debugging ROS applications, and other helpful development features. In this session, we will demonstrate how to build a simple ROS application from scratch using Hatchery and explore some of the features it provides. Attendees will also get a taste for how modern developer tools are implemented and learn some techniques for parsing and compiling domain-specific languages.
\section*{Description:} \\\\
%
This session will demonstrate a new IDE for ROS development. Primarily it consists of a rehearsed live demo illustrating some of the features which the IDE supports, as well as brief commentary about its implementation. Topics which will be covered in this tutorial:
%
\begin{enumerate}
    \item Creating a new ROS project in Hatchery
    \item Detecting and selecting a ROS distro
    \item Custom language support for ROS interface files
    \item Auto-completion and refactoring support
    \item Running and debugging a ROS application
\end{enumerate}

\noindent If time permits, attendees will also learn about parser-generators such as Xtext, Grammar-Kit and various tools for developing custom language support in an IDE of their own choice.
%
\noindent Please refer to this URL for further details about the project: \url{https://github.com/duckietown/hatchery/}
\end{document}
